\chapter*{Введение}
\addcontentsline{toc}{chapter}{Введение}

В условиях быстрого развития технологий и увеличения ассортимента товаров, компьютерные магазины сталкиваются с необходимостью оптимизации управления своими запасами и процессами продаж. Ручное ведение учета, которое использовалось ранее, стало неэффективным и требует значительных временных и трудовых ресурсов, что увеличивает вероятность ошибок и усложняет оперативное принятие решений. Внедрение системы управления ассортиментом на основе базы данных позволяет автоматизировать ключевые процессы, такие как отслеживание складских остатков, обработка транзакций. Такая система обеспечивает более точное управление запасами, снижает риск ошибок, ускоряет обработку информации и улучшает взаимодействие с клиентами. В результате, магазины могут значительно повысить свою производительность, снизить затраты времени и ресурсов, а также предоставить более высокий уровень обслуживания, что в конечном итоге способствует укреплению их конкурентных преимуществ на рынке. \cite{1}

\textbf{Цель данной работы} --- разработка базы данных для компьютерного магазина.

Чтобы достигнуть поставленной цели, требуется решить следующие задачи:
\begin{itemize}[label=--]
	\item проанализировать предметной области, требования к базе данных и приложению;
	\item определить основные роли пользователей;
	\item описать структуру базы данных, включая основные сущности и их атрибуты. Описать связи между сущностями и обеспечить целостность данных;
	\item разработать программное обеспечение для взаимодействия с базой данных, который позволит пользователям выполнять свои задачи;
	\item определить средства программной реализации;
	\item реализовать программное обеспечение для взаимодействия с базой данных, которое позволит пользователям выполнять свои задачи;
	\item провести экспериментальные замеры временных характеристик разработанного программного обеспечения.
\end{itemize}