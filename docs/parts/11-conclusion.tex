\chapter*{Заключение}
\addcontentsline{toc}{chapter}{Заключение}

В ходе  выполнения курсовой работы была проведена формализация задачи и данных, рассмотрены типы пользователей и требуемые функционалы. Также был проведен анализ существующих моделей баз данных и было решено использовать в данной работе реляционную СУБД. Спроектирована база данных и приложение для доступа к ней. Был спроектирован триггер, осуществляющий автоматический пересчет количества товаров на складе при добавления заказов. Представлены средства разработки программного обеспечения, выбор языка программирования и описан интерфейс программы.
Также рассмотрены примеры работы программы

В результате исследования было выяснено, что при использовании индексирования ускоряет выполнение запросов SELECT больше чем в 5 раз.

Данная программа может иметь следующие перспективы развития:
\begin{itemize}[label=---]
	\item анализ продаж: внедрение инструментов для анализа данных о продажах, тенденциях и предпочтениях клиентов;
	\item аналитические панели и отчеты: создание удобных интерфейсов для управления данными и генерации отчетов для аналитики;
\end{itemize}
